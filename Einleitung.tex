\chapter[Einleitung]{Einleitung\raisebox{.3\baselineskip}{\normalsize \footnotemark}}

Diese Studienarbeit behandelt die Konzipierung und Implementierung eines
digitalen Funktionsgenerators in der Hardwarebeschreibungssprache VHDL.
Ein Funktionsgenerator ist ein elektronisches Bauteil, das in der Lage ist,
verschiedene Spannungsverläufe an seinem Ausgang auszugeben. Diese
Spannungsverläufe entsprechen einer mathematischen Funtkion. Z. B. kann ein
Funktionsgenerator genutzt werden, um ein Rechteck-Signal mit einer bestimmten
Frequenz auszugeben, dass dann als Auslöser für eine Kamera fungiert, sodass die Kamera Bilder in einem bestimmten zeitlichen Abstand aufnimmt.
Im Normalfall würde ein Funtkionsgenerator als Digitalschaltung in einen Chip
integriert oder auf eine Platine gelötet werden. Einen anderen Ansatz zum Bau
von digitalen Schaltungen bieten ``Free Programmable Gate Arrays'', kurz FPGA.
Auf diesen ICs befinden sich verschiedene Bausteine die durch Anlegen einer
Programmierspannung miteinander verknüpft werden können. Somit ist es möglich,
verschiedenste Schaltungen auf demselben IC zu verwirklichen.
Die Schaltungen können mithilfe einer Beschreibungssprache designed werden.
Eine dieser Sprachen ist VHDL (``\textbf{V}ery \textbf{H}ighspeed Hardware \textbf{D}escription \textbf{L}anguage''), welche in dieser Studienarbeit verwendet werden soll. 

