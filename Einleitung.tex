\chapter[Einleitung]{Einleitung}

Diese Studienarbeit behandelt die Konzipierung und Implementierung eines digitalen Funktionsgenerators in der Hardwarebeschreibungssprache VHDL.
Ein Funktionsgenerator ist ein elektronisches Bauteil, das in der Lage ist, verschiedene Spannungsverläufe an seinem Ausgang auszugeben.
Diese können dann an andere Bauteile angeschlossen und so technisch genutzt werden.
Beispielsweise kann ein Funktionsgenerator eingesetzt werden, um ein Rechteck- oder Sinussignal auszugeben.
Der praktische Nutzen könnte dann die Aktivierung eines Kameraauslösers in einer bestimmten Frequenz oder die Helligkeitssteuerung einer LED sein. \\
Ein Funtkionsgenerator kann als Digitalschaltung in einen Chip integriert oder auf eine Platine gelötet werden, er könnte aber auch als Programm eines Universalrechners laufen.
Softwarelösungen verfügen nicht über das hohe Maß an Effizienz und die Geschwindigkeit einer integrierten Schaltung.
Diese wiederum sind nur in großer Stückzahl rentabel herstellbar.
Von Hand gelötete Schaltungen lassen sich nicht einfach reproduzieren und sind nicht besonders kompakt.
Einen Mittelweg zwischen diesen Zielkonflikten bieten sogenannte ``\textbf{F}ree \textbf{P}rogrammable \textbf{G}ate \textbf{A}rrays'', kurz FPGA.
Auf diesen ICs befinden sich verschiedene Bausteine die durch Anlegen einer Programmierspannung miteinander verknüpft werden können.
Somit ist es möglich, verschiedenste Schaltungen auf demselben IC umzusetzen. \\
Die Schaltungen können mithilfe einer Beschreibungssprache designed werden.
Eine dieser Sprachen ist VHDL (``\textbf{V}ery \textbf{H}ighspeed Hardware \textbf{D}escription \textbf{L}anguage''), welche in dieser Studienarbeit verwendet werden soll.
Die Implementation auf einem FPGA mittels Beschreibungssprache bietet den Vorteil guter Reproduzierbarkeit bei vergleichsweise hoher Konfigurierbarkeit und Effizienz. \\
Bei dem in diesem Projekt verwendeten FPGA handelt es sich um den Artix 7 von Xilinx.
Dieser ist in das Entwicklungsboard Basys 3 des Herstellers digilent eingebettet \cite{digilent2016}.
Es verfügt über diverse Peripheriemodule wie eine UART Schnittstelle, einen micro-USB Anschluss oder einen VGA Ausgang. \\
In dieser Arbeit wird zunächst der konzeptuelle Aufbau des Funktionsgenerators erläutert, dann werden die einzelnen Komponenten, aus denen er besteht, sowie die interne Taktung der Komponenten erläutert.
Schließlich wird noch ein Funktionstest durchgeführt, bei dem die theoretisch erwarteten Resultate den tatsächlichen gegenübergestellt werden. \\
Sämtlicher Quellcode findet sich im GitHub-Repository unter folgender URL: \\
\href{https://github.com/markushart/studienarbeit_function_generator.git}{\code{https://github.com/markushart/studienarbeit\_function\_generator.git}}