\chapter{Ausblick}
Neben den im Fazit genannten weiteren Untersuchungen gibt es noch mehrere Aspekte des Funktionsgenerators, bei dem sich weitere Forschungs- und Entwicklungsarbeit lohnen könnte.\\
Durch die Implementation der Funktionskomponenten als Bausteine ist es relativ leicht, die möglichen Funktionsformen zu erweitern.
Z.B. könnte noch eine Sinusfunktion hinzugefügt werden.
Theoretisch könnte auch ein nicht-periodisches Signal, wie ein zeitversetzter Sprung hinzugefügt werden.\\
Außerdem kann die UART-Schnittstelle noch um weitere Features ergänzt werden.
Es wäre beispielsweise denkbar, die gegenwärtigen Funktionsparameter über den Transmitter an den User zurück zu schicken, sodass dieser sie leichter kontrollieren kann.
Dazu müsste auch die Konfigurationsschnittstelle um Befehle zur Rückgabe der Funktionsparameter erweitert werden. \\
Darüber hinaus könnte auch noch der zweite Kanal auf dem DAC-Konverter genutzt werden um zwei verschiedene Signale gleichzeitig auszugeben.
Zusätzlich könnten die Zusatzfunktionen des Konverters genutzt werden, um den Konverter in den Energiesparzustand zu versetzen.
Es wäre denkbar, einen entsprechenden Befehl in die Konfigurationsschnittstelle einzubauen. \\
Abschließend fehlt auch noch der Test in einer realen Anwendung, z.B. einem Trigger für eine Fotokamera, die in einer bestimmten Frequenz Bilder aufnehmen soll.