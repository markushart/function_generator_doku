\documentclass[11pt]{beamer}
\useoutertheme{infolines}
\setbeamertemplate{headline}

\useinnertheme{default}
\usecolortheme{beaver}


\usepackage[utf8]{inputenc}
\usepackage[english]{babel}
\usepackage{amsmath}
\usepackage{amsfonts}
\usepackage{amssymb}
\usepackage{graphicx}
\graphicspath{{../imgs/}}
\usepackage{hyperref}
\urlstyle{same}

\usepackage[
backend=bibtex,
sortcites=false,
style=numeric,
firstinits=true,
uniquename=init,
isbn=false,
pagetracker=false,
maxbibnames=50,
maxcitenames=3,
autocite=inline,
block=space,
backref=false,
backrefstyle=three+,
date=short,
url=false,
doi=true,
eprint=false,
sorting=none
]{biblatex}
\addbibresource{refs.bib}

\usefonttheme[onlymath]{serif}


\author{Markus Hartlage}
\title[Entwicklung und Implementierung eines digitalen Funktionsgenerators in VHDL]{Entwicklung und Implementierung eines digitalen Funktionsgenerators in VHDL}
%\setbeamercovered{transparent} 
%\setbeamertemplate{navigation symbols}{} 
\titlegraphic{\includegraphics[width=3cm]{logo.png}}
\institute{FH-Bielefeld} 
%\date{} 
%\subject{}
\begin{document}

\begin{frame}
\titlepage
\end{frame}

%\begin{frame}
%\tableofcontents
%\end{frame}

% Konzept
% Funktionskomponenten
% Konfiguration - hier kann ich vielleicht das python Interface zeigen
% Funktionstests
% Fazit

\section{Konzept}
\begin{frame}{Konzept}

\begin{enumerate}
	\item Stance phase
	\begin{itemize}
		\item Movement of the foot on the ground
		\item Foot moves backwards (respective to walking direction)
		\item Backwards movement forces the robot body to move froward
	\end{itemize}
	\item Swing phase
	\begin{itemize}
		\item Movement of the foot through free space without ground contact
		\item Foot moves forwards 
		\item Start- and endpiont are on the ground for transition from and to stance movement
	\end{itemize}
\end{enumerate}
\end{frame}

\section{Swing trajectory constraints}
\begin{frame}{Swing trajectory constraints}
\begin{itemize}
	\item General swing phase constraints
	\begin{itemize}
		\item Start position
		\item End position
	\end{itemize}
	\pause
	\item Constraints due to the robot design
	\begin{itemize}
		\item Working space for each joint of a leg
		\item Minimal required swing height
		\item Acceleration and speed of joint drives
	\end{itemize}
	\pause
	\item Environmental constraints
	\begin{itemize}
		\item Position and size of obstacles
	\end{itemize}
\end{itemize}
\end{frame}

\section{Trajectory planning methods}
\begin{frame}[allowframebreaks]{Trajectory planning methods --  Global approaches}
	\begin{itemize}
		\begin{columns}
		\begin{column}{0.6\textwidth}
		\item Roadmap
		\begin{itemize}
			\item Corners of obstacles are connected by straigth lines to form the shortest path along the obstacles
			\item Obstacles must be known as polygons
		\end{itemize}
		\end{column}
		\begin{column}{0.25\textwidth}
		\begin{figure}
		% \includegraphics[scale=.2]{img/latombe-roadmap}{\tiny\cite{latombe}}
		\end{figure}
		\end{column}
		\end{columns}
		
		\begin{columns}
		\begin{column}{0.6\textwidth}
		\item Cell decomposition
		\begin{itemize}
			\item Free space divided into cells
			\begin{itemize}
				\item Exact decomposition
				\item Approximate decomposition
			\end{itemize}
			\item Obstacles must be known
			\item Neighbouring cells are represented in a graph
			\item A path in the graph represents one possible trajectory
		\end{itemize}
		\end{column}
		\begin{column}{0.25\textwidth}
		\begin{figure}
		% \includegraphics[width=\textwidth]{img/latombe-celldecomposition}{\tiny\cite{latombe}}
		\end{figure}
		\end{column}
		\end{columns}
		
		\begin{columns}
		\begin{column}{0.5\textwidth}
		\item Potential field
		\begin{itemize}
			\item Free space is discretized in a mesh
			\item Obstacle points and start point are weighted repelling
			\item Goal point is weighted attracting
			\item Trajectory is defined by a gradient descent
			\item Problem of getting trapped in local minima
			\item Obstacles must be known
		\end{itemize}
		\end{column}
		\begin{column}{0.4\textwidth}
		\begin{figure}
		% \includegraphics[scale=.2]{img/latombe-potentialfield}{\tiny\cite{latombe}}
		\end{figure}
		\end{column}
		\end{columns}
		
	\end{itemize}
\end{frame}

\begin{frame}{Trajectory planning methods -- Local approaches}
	\begin{itemize}
		\item Self defined trajectory
		\item Selecting additional constraints (e.g. approach angle)
		\item Defining via points
		\item Connencting via points by different methods
		\begin{itemize}
			\item Linear
			\item Linear with parabolic blends
			\item (Cubic) splines \textit{(one proposed approach by \textcite{zeng})}
			\item Elliptic \textit{(approach used by \textcite{paskarbeit} on HECTOR)}
		\end{itemize}
		\item Collisions with obstacles must be handled or avoided
		\begin{figure}		
		% \includegraphics[width=0.3\textwidth]{img/zeng-splines} {\tiny\cite{zeng}}
		% \includegraphics[width=.5\textwidth]{img/paskarbeit-ellipses}{ \tiny\cite{paskarbeit}}
		\end{figure}
		
	\end{itemize}
\end{frame}

\begin{frame}{Deciding for one approach}
\begin{itemize}
\item In order to implement a trajectory generator one of these methods has to be chosen
\item This decision is highly dependent on the robot it has to genereate trajectories for
\item If the robot has a notion of it's surrounding space, a global approach is possible
\item Otherwise a local approach has to be used and collisions have to be handled
\item Since the global approaces use a discretisation of the space and therefore have to compute many possible paths to choose one for each step
\item Using a naive local approach requires little computation and allows reusage of a path 
\item A notion of the surrounding space in combination with a local approach can be used to minimize collisions by coosing optimal goal positions

\end{itemize}
\end{frame}

\begin{frame}{Sources}
\printbibliography[title={Sources},heading=bibintoc]
\end{frame}

\end{document}