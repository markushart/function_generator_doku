\chapter[Konzept]

Im folgenden soll das Konzept des Funktionsgenerators erläutert werden und
seine Funktionsweise erklärt werden. 
 
\section{was kann das gerät}
der generator kann 4 verschiedene Funktionen ausgeben,

\begin{enumerate}
  \item Konstante 
    Ein konstanter Wert liegt am Ausgang an.
  \item Rechteck-Funktion
    Der Wert wechselt zwischen High- und Low-Pegel in der Frequenz $f$.
    Der Anteil der Zykluszeit $T$, in dem der Ausgang auf High ist, wird über
    den dutycycle eingestellt.
  \item Zick-Zack-Funktion
    Der Analogwert steigt vom Low-Pegel bis zum High-Pegel linear
    an, erreicht er den High-Pegel, fällt der Analogwert wieder kontinuierlich
    auf Low ab. Somit schwankt der Pegel mit der Frequenz $f$.
  \item Rampen-Funktion 
    Der Analogwert wächst, wie bei der Zick-Zack-Funktion, linear bis auf High
    an, dann fällt er aber auf Low zurück. Alternativ kann die Rampe auch vom
    High-Pegel her abfallen und bei Erreichen von Low wieder auf High zurück springen.
\end{enumerate}

\begin{figure}
  \centering
  \begin{subfigure}[b]{0.2\textwidth}
      \centering
    \includesvg[width=6cm]{constant_function}
    \label{constant_function_graph}
  \end{subfigure}
  \hfil
  \begin{subfigure}[b]{0.2\textwidth}
    \centering
    \includesvg[width=6cm]{square_function}
    \label{square_function_graph}
  \end{subfigure}
  \vfil
  \begin{subfigure}[b]{0.2\textwidth}
    \centering
    \includesvg[width=6cm]{zigzag_function}
    \label{zigzag_function_graph}
  \end{subfigure}
  \hfil
  \begin{subfigure}[b]{0.2\textwidth}
    \centering
    \includesvg[width=6cm]{ramp_function}
    \label{ramp_function_graph}
  \end{subfigure}
  \hfil
\end{figure}
