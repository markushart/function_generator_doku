\chapter{Komponenten}
In diesem Kapitel soll der Aufbau des Funktionsgenerators anhand seiner
digitaltechnischen Komponenten erläutert werden.

\section{Arithmetik}

\subsection{Zähler}
\subsection{Teiler}
Während beim Multiplizieren, Addieren und Subtrahieren auf Standardfunktionen
zurückgegriffen wird, wird für die Division zweier Binärzahlen eine eigene
Komponente entworfen. Die Gründe hierfür werden in \cref{Comp:Func:ZigZag}
erläutert. 

\section{Takterzeugung}
\subsection{Clock-Enable}

\section{Funktionen}   \label{Comp:Func}
\subsection{Konstante}   \label{Comp:Func:Const}
\subsection{Rechteck}   \label{Comp:Func:Square}
\subsection{Zick-Zack}  \label{Comp:Func:ZigZag}

\subsection{Rampe} \label{Comp:Func:Ramp}

\section{Konfigurationsschnittstelle}
Die Konfigurationsschnittstelle \code{CONFIG\_INTERFACE} besteht aus einer
UART-Schnittstelle, über die der Datenaustausch zwischen Benutzer und
Funktionsgenerator erfolgt, sowie der Instruktionsauswertung, die die
empfangenen UART-Signale in Konfigurationsbefehle übersetzt.
\subsection{UART-Schnittstelle}
Die UART-Schnittstelle beruht auf dem
\textbf{U}niversal-\textbf{A}synchronous-\textbf{R}ceiver-\textbf{T}ransmission-Protokoll.
Das Protokoll ermöglicht es, byteweise serielle Daten zu verschicken und zu
empfangen. Hierfür reichen zwei Drähte aus, die jeweils eins der beiden Signale
RX (Receive) und TX (Transmit) transportieren. Zum Start der Kommunikation wird
die RX Leitung vom Sender von high auf low gezogen, sodass der Empfänger anfängt
die nachfolgenden acht Bits zu einem Byte zusammenzusetzen. Anschließend muss
mindestens ein Stop-Bit folgen, bei dem die Receive Leitung des Empfängers auf
High liegt. Darauf kann je, nach Implementierung, noch ein Stop-Bit sowie
ein Paritätsbit folgen. Da es zwischen dem Sender und Empfänger kein synchrones
Taktsignal gibt, ist es wichtig, dass ihre Sende- und Empfangsfrequenz gleich
ist. Diese Frequenz ist die sogenannte Baudrate. Im Funktionsgenerator ist sie
auf 115200 Bits / s festgelegt. Die im Generator verwendete Schnittstelle wurde,
um den Arbeitsaufwand zu verringern, aus einer Vorlage übernommen (Quelle). Sie
beinhaltet sowohl eine Empfänger- als auch eine Sender-Komponente. Es gibt
folgende Eingangssignale:
\begin{enumerate}
\item \code{CLK}: Eingang, gibt die Taktfrequenz der Komponente vor
\item \code{CE}: Eingang, ''chip-enable``-Signal, aktiviert die Komponente wenn es auf
  low gesetzt wird
\item \code{reset}: Eingang, die Schnittstelle wird auf den Initialisierungszustand
  zurückgesetzt und die aktuelle Übertragung bzw. aktuelle Empfangsprozesse
  werden abgebrochen.

\item \code{tx}: Ausgang, Das von der Schnittstelle versendete TX-Signal 
  \item \code{tx\_start}: Eingang, wenn \code{tx\_start} auf high gesetzt wird, wird mit der
    Übertragung von \code{data\_in} begonnen
\item \code{data\_in}: Eingang, ein 8-Bit breites Signal, dass das zu versendende Byte enthält.

\item \code{rx}: Eingang, das von der Schnittstelle empfangene RX-Signal
\item \code{data\_out}: Ausgang, ein 8-Bit breites Signal, dass das zuletzt von der
  Schnittstelle empfangene Byte beinhaltet.
\item \code{rx\_uart\_rdy}: Ausgang, dieses Signal zeigt an, wenn ein komplettes Byte
  empfangen wurde und bereit ist, gelesen zu werden.
\end{enumerate}
  
\subsection{Instruktionsauswertung}

\section{DAC-Konverter}
\subsection{DAC-Kanal}

