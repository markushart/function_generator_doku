\chapter{Funktionstest}
Nach der Implementierung des zuvor geschilderten Konzepts wurde ein Funktionstest durchgeführt.
Dazu werden zunächst die theoretischen Grenzen der Konfiguration getestet und danach die tatsächlichen Ausgabewerte mit den theoretischen verglichen.
Besonders interessant ist hierbei, in welchem Frequenzbereich der Generator zuverlässig funktioniert.

\section{theoretische Limitierungen}
Um den praktischen Nutzen des Funktionsgenerators einzuschätzen, werden die Grenzen der einstellbaren Frequenz, die Auflösung und der Spannungsbereich berechnet.
Für die Spannung ergeben sich diese Werte aus dem Datenblatt des eingesetzten DACs, dieser kann in einem Bereich von 0 bis 5V eingesetzt werden \cite{PmodDA2}.
Für den Betrieb auf dem Basys 3 Board läuft der Spannungsbereich von 0 bis 3,3 V, da hier die Ausgangsspannung des Boards der limitierende Faktor ist.
Nachfolgend werden noch der Frequenzbereich und die Auflösung bestimmt.

\subsection{Frequenzbereich}
Der Systemtakt $f_{sys}$ des Generators beträgt 100 MHz.
Der Frequenzbereich der Funktionen, die er ausgeben kann, reicht von ca. 0,0931 Hz bis zu ca. 781 kHz.
Die Frequenz $f_{count}$ ist die Frequenz mit der der interne Zähler der Funktionsbausteine hochzählt und beträgt ein 64tel des Systemtakts $f_{sys}$ (siehe \cref{test:theo:freq:fcount}).
Die minimale Frequenz $f_{min}$ errechnet sich aus dem maximalen Zählstand, der wiederum abhängig ist von seiner Bitbreite \code{clk\_width} (siehe \cref{test:theo:freq:fmin}).
Die maximale Frequenz ergibt sich aus dem Shannon'schen Abtasttheorem, nach dem die minimale Abtastfrequenz eines Signals doppelt so groß sein muss, wie die Frequenz des abgetasteten Signals \cite(blabla).
In diesem Fall entspricht die Abtastfrequenz $f_{count}$ und das abgetastete Signal dem Ausgangssignal, woraus folgt, dass das ausgehende Signal nur halb so groß sein kann, wie $f_{count}$ (siehe \cref{test:theo:freq:fmin}).

\begin{align}
  f_{count} &= \frac{f_{sys}}{64} \label{test:theo:freq:fcount}\\  
  f_{max} &= \frac{f_{count}}{2}  \label{test:theo:freq:fmax}\\ 
  f_{min} &= \frac{f_{count}}{2^{clk\_width} - 1} \label{test:theo:freq:fmin}
\end{align}

\subsection{Auflösung}
Die Auflösung des analogen Ausgangssignals hängt sowohl von der Geschwindigkeit ab, mit der das digitale Signal analogisiert werden kann, als auch von der maximalen Anzahl digital darstellbarer Werte.
Überschreitet die Frequenz des analogen Signals $f$ die Grenzfrequenz $f_{grenz}$, so fällt die Auflösung $R$ reziprok zu $f$ ab (siehe \cref{test:theo:res:plot}).
Oberhalb von $f_{grenz}$ ist die Auflösung durch die Bitbreite des ADCs begrenzt.
Da es sich um einen 12-Bit Wandler handelt, beträgt die Anzahl darstellbarer Werte und damit auch die höchste Auflösung $2^{12} = 4096$.
Diesen Wert nimmt die Auflösung an, wenn die Frequenz kleiner als $f_{grenz}$ ist (siehe \cref{test:theo:freq:fgrenz}).

\begin{align}
  R &= \begin{cases}
    4095               & f \leq f_{grenz}                      \\
    \frac{f_{count}}{f} & f > f_{grenz} = \frac{f_{count}}{4095}
        \end{cases} \label{test:theo:freq:fgrenz}
\end{align}

Konkret bedeutet dass für den Funktionsgenerator, dass bei einer Amplitude von $U_{SS}=3,3V$ und einer Frequenz von $f=100kHz$, die Auflösung $4,54V^{-1}$ statt der maximalen Auflösung von $1241V^{-1}$ beträgt, das heißt, dass bei 100kHz 4,54 Werte statt 1241 Werte pro Volt gesampelt werden.

\begin{figure}[h] \centering
    \begin{tikzpicture}
      \begin{loglogaxis}
        [xlabel=$f\:in\:Hz$,
        ylabel=$R(f)$,
        ymin=1,
        ymax=10000,,
        width=\textwidth,
        height=0.5\textwidth,
        grid=major]
        \addplot[color=black, domain=0.09313:381.56] {4095};
        \addplot[color=black, domain=381.56:781250] {100000000 / (64*\x)};
        % help lines and nodes:
        \addplot[color=red] coordinates{(381.56, 1)(381.56, 10000)};
        \node[label={[rotate=90]above:$f_{grenz}$}] at (381.56, 100);
        \node at (381.56, 4095) {\textbullet};

        \addplot[color=red] coordinates{(0.09313, 1)(0.09313, 10000)};
        \node[label={[rotate=90]above:$f_{min}$}] at (0.09313, 100);
        \node at (0.09313, 4095) {\textbullet};

        \addplot[color=red] coordinates{(781250, 1)(781250, 10000)};
        \node[label={[rotate=90]above:$f_{max}$}] at (781250, 100);
        \node at (781250, 2) {\textbullet};
      \end{axis}
    \end{tikzpicture}
 \caption{Doppelt logarithmisches Diagramm der Auflösung $R(f)$ über den Frequenzbereich $f$. Die Auflösung bleibt konstant bei 4095 1/tick bis sie schließlich bei $f_{grenz}$ anfängt zu sinken.} \label{test:theo:res:plot}
\end{figure}

\section{reales Verhalten}
Nun soll das reale Verhalten des Funktionsgenerators untersucht werden.
Dazu wird das digitale Oszilloskop Analog Discovery 2 von digilent benutzt.

