\chapter{Fazit}
Es wurde erfolgreich ein digitaler Funktionsgenerator auf dem Basys 3 Board implementiert.
Dieser kann vier verschiedene Funktionen ausgeben und über eine UART Schnittstelle konfiguriert werden. \\
Das dem Generator zugrundeliegende Konzept und das Zusammenspiel der Einzelkomponenten wurde erläutert.
Auch das interne Clock-Management wurde geschildert. \\
Die theoretischen Grenzen der Auflösung und der Frequenz wurden aufgezeigt. 
Darüber hinaus konnte seine Funktionstüchtigkeit in dem Frequenzbereich 0,1 Hz bis 10 kHz bewiesen werden.
Alle Frequenzen unterhalb von $f_{min}$ führen zu einer höheren Ausgangsfrequenz als der gewünschten.
Zwischen 10 kHz und $f_{max}$ wird das Ausgangssignal unsauber und oberhalb von $f_{max}$ ist eine zuverlässige Ausgabe nicht mehr möglich.
Warum das Signal oberhalb von 10 kHz derart ungenau wird, konnte teilweise erklärt werden.
Hier sollten jedoch weitere Untersuchungen angestrebt werden, um die Fehlerursache vollständig zu klären.\\
Die Ausgangsspannung entspricht nicht ganz der eingestellten Spannung.
Eine bessere Referenzspannung könnte aber für einen saubereren Ausgangspegel sorgen.
Allerdings wurde die Korrektheit des Ausgangspegels bis jetzt nur für die Spannung 0 und 3,3 V untersucht.
Um ein vollständigeres Bild der Genauigkeit zu bekommen, sind weitere Nachforschungen nötig.\\
Es wurde gezeigt, dass die Konfiguration per UART-Schnittstelle funktioniert.
Außerdem wurde der Befehlssatz des Funktionsgenerators dokumentiert, sodass andere Entwickler und Nutzer mit dem Generator interagieren können.
